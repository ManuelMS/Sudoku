% ======== Plantilla para elaboracion del informe final ========
% version: 1.0  - Agosto/2017
% Autores:
% Dennis Rodríguez - Luis Vásquez - Walter Chaves - Manuel Masís
% Comentarios y sugerencias en el foro del taller:
% https://eva.fing.edu.uy/course/view.php?id=307
% ========================================================
%
% ----- INSTRUCCIONES -----
% Abrir archivos README.txt incluido en *.zip
% -------------------------
%
% ----- LICENCIA -----
% Este trabajo es distribuido bajo la licencia LaTeX Project Public License
% Puede y DEBE ser usado, modificado y distribuido gratuitamente.

% --- Clase de archivo ---
% Con este comando se define que tipo de documento vamos a hacer
% en este caso un articulo, en una hoja a4 y con tamanio de fuente 12pt.
\documentclass[a4paper,openright,12pt]{report}

% A partir de aqui se definiran ciertos paquetes o funciones a ser utilizadas
% dentro del documento.

% --- idioma ---
% El paquete babel sirve para separar correctamente las palabras en muchos idiomas,
%aquí esta configurado con español (spanish). Tambien sirve para que el Indice tenga 
%título "Indice" en lugar de "Contents".
\usepackage[english]{babel} % espanol

% --- codificacion de archivo ---
% El paquete inputenc sirve para que el compilador pueda interpretar los acentos
%en español de forma estandar, dependiendo del sistema operativo hay que darle una
%configuracion diferente:

%Para windows se escribe:
%\usepackage[ansinew]{inputenc}
%\usepackage[latin1]{inputenc} % acentos sin codigo
\usepackage[utf8]{inputenc}

% --- Gráficos ---
% El paquete graphicx sirve para controlar figuras con \includegraphics.
\usepackage{graphicx} % graficos

% --- Hipervínculos ---
% paquete para marcar los hiperví­nculos en i­ndice y referencias
\usepackage{hyperref}

% Para agregar al índice las refencias
\usepackage[nottoc,notlot,notlof]{tocbibind}

% ===== Encabezado =====
%Esta es una posible configuración para el encabezado. 
%Si se comentan estas dos lineas no habrá encabezado y la numeración de página aparecerá abajo de cada hoja. En la página donde se llame a \maketitle no se coloca encabezado.
\pagestyle{myheadings}
\markright{UNA \hfill Informe final en \LaTeX \hfill}
% ======================


%% ===== Ajuste layout pagina =====
% define el ancho del texto en la hoja
\setlength{\textwidth}{155mm}
% define el alto del texto en la hoja
\setlength{\textheight}{210mm}
% los márgenes pueden ser editador con
\oddsidemargin=-.25cm
%% ================================

%Aqui comienzan los datos del trabajo. El comando \date{\today} asigna la fecha
%en que se compila como la fecha del trabajo, tambien se puede escibe directamente
%Ejemplo: \date{5 de setiembre de 2012}.

\begin{document}

\begin{titlepage}

\begin{center}
\vspace*{-1in}
\begin{figure}[htb]
\begin{center}
%Aca se inserta el logo de la U
\includegraphics[width=6cm]{./logo}
\end{center}
\end{figure}
UNIVERSIDAD NACIONAL\\
\vspace*{0.15in}
FACULTAD DE CIENCIAS EXACTAS Y NATURALES\\
\vspace*{0.15in}
ESCUELA DE INFORMATICA\\
\vspace*{0.6in}
\begin{Large}
CURSO: PARADIGMAS DE PROGRAMACION\\
\end{Large}
\vspace*{0.2in}
\begin{Large}
\textbf{INFORME FINAL PROYECTO:}\\
Algoritmos de Generación y Resolución de Sudokus\\
\end{Large}
\vspace*{0.5in}
\begin{large}
Profesor:\\
Carlos Loria Saenz\\
\end{large}
\vspace*{0.3in}
\begin{large}
Integrantes:\\
Dennis Rodríguez\\
Luis Vásquez\\
Manuel Masís\\
Walter Chaves\\
\end{large}
\vspace*{0.3in}
\rule{80mm}{0.1mm}\\
\vspace*{0.1in}
\begin{large}
%Aca se inserta el logo de pie de pagina
Año 2017\includegraphics[width=4cm]{./logoP}
\date{\today}
\end{large}
\end{center}

\end{titlepage}

% índice
\tableofcontents

% para separar la carátula del texto introducimos un salto de pagina
\newpage

% definimos una sección con el comando section

\section{Introducción}
%
Acá vamos a agregar la introducción respecto al tema que nos atañe, el cual para nuestro caso se refiere a
la aplicación de algoritmos combinatorios para la generación y resolución de SUDOKUS.

\section{Marco Teórico}
%
En esta sección presentaríamos un marco teórico del contenido del trabajo, generalmente debe ser lo mínimo indispensable para obtener los resultados y concluir.

\section{Desarrollo}
%
El desarrollo sería una de las partes centrales de un informe o artículo, en este caso presentaremos ejemplos de creación de fórmulas matemáticas con \LaTeX.

\subsection{Fórmulas matemáticas}
Hay varias maneras de insertar una fórmula matemática en un trabajo. Primero y más fácil es en la misma línea, como por ejemplo: $E=mc^{2}$. También se puede hacer en una línea aparte, sin numerar de esta forma $$E=mc^{2}$$ después se pueden poner en una línea aparte y numerarlas, de esta forma: 
\begin{equation}\label{ec:Ein}
  E=mc^{2}
\end{equation} %
%
%cuando se escribe en el entorno \say{equation} es conveniente no dejar un renglón en blanco entre el texto y %el entorno, así LaTeX dejará el espacio adecuado a cada fórmula. Nótese que se utiliza aquí el paquete \texttt{dirtytalk}, en donde \say{se pueden colocar citas \say{dentro de} otras citas}. Se pueden hacer %referencias a todas las etiquetas creadas, por ejemplo a la ecuación~\eqref{ec:Ein} o al \autoref{tab:cuadro_2}, incluso a la sección~\ref{ej_fonts}.


\subsection{Imágenes}
%
La \autoref{fig:optimo} es una figura de ejemplo, para un formato de la figura jpg, es necesario compilar el documento con el compilador PDFLaTeX, para formatos de figura eps, se puede compilar en Latex.
\begin{figure}[h!]
\centering
%\includegraphics[width=1\textwidth]{figuras/optimo_primal}
\label{fig:optimo}
\caption{}
\end{figure}

\section{Conclusiones}
%
Aquí irían las conclusiones del artículo.  Tal vez se quieran mostrar conclusiones de forma esquemática como items...
%
\begin{itemize}
  \item Conclusión numero 1... 
  \item otra conclusión.
\end{itemize}

o también enumerándolas

\begin{enumerate}
  \item item primero...
  \item item segundo...
\end{enumerate}


\subsection{Citación de bibliografía}
%
Es posible citar artículos utilizados como referencia utilizando la función ``cite''. Por ejemplo citamos el libro \cite{lamport94} utilizando la etiqueta lamport94 que fue definida previamente al final del archivo latex como se puede ver en el código de este ejemplo. %
%
Se pueden citar tantos artículos como se quiera siempre que estén incluidos en el archivo \cite{otraetiqueta}. Existen otras formas mas complejas de citar en trabajos grandes como tesis o libros, utilizando archivos .bib . % ver http://en.wikibooks.org/wiki/LaTeX/Bibliography_Management

\end{document}